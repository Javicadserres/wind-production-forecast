%----------
%   IMPORTANTE
%----------

% Esta plantilla está basada en las recomendaciones de la guía "Trabajo fin de Máster: Escribir el TFM", que encontrarás en http://uc3m.libguides.com/TFM/escribir
% contiene recomendaciones de la Biblioteca basadas principalmente en estilos APA e IEEE, pero debes seguir siempre las orientaciones de tu Tutor de TFM y la normativa de TFM para tu titulación.



% ESTA PLANTILLA ESTÁ BASADA EN EL ESTILO APA


%----------
%	CONFIGURACIÓN DEL DOCUMENTO
%----------

\documentclass[12pt]{report} % fuente a 12pt

% MÁRGENES: 2,5 cm sup. e inf.; 3 cm izdo. y dcho.
\usepackage[
a4paper,
vmargin=2.5cm,
hmargin=3cm
]{geometry}

% INTERLINEADO: Estrecho (6 ptos./interlineado 1,15) o Moderado (6 ptos./interlineado 1,5)
\renewcommand{\baselinestretch}{1.15}
\parskip=6pt

% DEFINICIÓN DE COLORES para portada y listados de código
\usepackage[table]{xcolor}
\definecolor{azulUC3M}{RGB}{0,0,102}
\definecolor{gray97}{gray}{.97}
\definecolor{gray75}{gray}{.75}
\definecolor{gray45}{gray}{.45}

% Soporte para GENERAR PDF/A --es importante de cara a su inclusión en e-Archivo porque es el formato óptimo de preservación y a la generación de metadatos, tal y como se describe en http://uc3m.libguides.com/ld.php?content_id=31389625. 

% En la plantilla incluimos el archivo OUTPUT.XMPDATA. Puedes descargar este archivo e incluir los metadatos que se incorporarán al archivo PDF cuando compiles el archivo memoria.tex. Después vuelve a subirlo a tu proyecto.  
\usepackage[a-1b]{pdfx}

% ENLACES
\usepackage{hyperref}
\hypersetup{colorlinks=true,
	linkcolor=black, % enlaces a partes del documento (p.e. índice) en color negro
	urlcolor=blue} % enlaces a recursos fuera del documento en azul

% EXPRESIONES MATEMÁTICAS
\usepackage{amsmath,amssymb,amsfonts,amsthm}

% Codificación caracteres
\usepackage{txfonts} 
\usepackage[T1]{fontenc}
\usepackage[utf8]{inputenc}

% Definición idioma español
\usepackage[spanish, es-tabla]{babel} 
\usepackage[babel, spanish=spanish]{csquotes}
\AtBeginEnvironment{quote}{\small}

% diseño de PIE DE PÁGINA
\usepackage{fancyhdr}
\pagestyle{fancy}
\fancyhf{}
\renewcommand{\headrulewidth}{0pt}
\rfoot{\thepage}
\fancypagestyle{plain}{\pagestyle{fancy}}

% DISEÑO DE LOS TÍTULOS de las partes del trabajo (capítulos y epígrafes o subcapítulos)
\usepackage{titlesec}
\usepackage{titletoc}
\titleformat{\chapter}[block]
{\large\bfseries\filcenter}
{\thechapter.}
{5pt}
{\MakeUppercase}
{}
\titlespacing{\chapter}{0pt}{0pt}{*3}
\titlecontents{chapter}
[0pt]                                               
{}
{\contentsmargin{0pt}\thecontentslabel.\enspace\uppercase}
{\contentsmargin{0pt}\uppercase}                        
{\titlerule*[.7pc]{.}\contentspage}                 

\titleformat{\section}
{\bfseries}
{\thesection.}
{5pt}
{}
\titlecontents{section}
[5pt]                                               
{}
{\contentsmargin{0pt}\thecontentslabel.\enspace}
{\contentsmargin{0pt}}
{\titlerule*[.7pc]{.}\contentspage}

\titleformat{\subsection}
{\normalsize\bfseries}
{\thesubsection.}
{5pt}
{}
\titlecontents{subsection}
[10pt]                                               
{}
{\contentsmargin{0pt}                          
	\thecontentslabel.\enspace}
{\contentsmargin{0pt}}                        
{\titlerule*[.7pc]{.}\contentspage}  

% DISEÑO DE TABLAS y FIGURAS
\usepackage{multirow} % permite combinar celdas 
\usepackage{caption} % para personalizar el título de tablas y figuras
\usepackage{floatrow} % utilizamos este paquete y sus macros \ttabbox y \ffigbox para alinear los nombres de tablas y figuras de acuerdo con el estilo definido.
\usepackage{array} % con este paquete podemos definir en la siguiente línea un nuevo tipo de columna para tablas: ancho personalizado y contenido centrado
\newcolumntype{P}[1]{>{\centering\arraybackslash}p{#1}}
\DeclareCaptionFormat{upper}{#1#2\uppercase{#3}\par}
\usepackage{graphicx}
\graphicspath{{imagenes/}} % ruta a la carpeta de imágenes

% Diseño de tabla para ciencias sociales y humanidades
\captionsetup*[table]{
	justification=raggedright,
	labelsep=newline,
	labelfont=small,
	singlelinecheck=false,
	labelfont=bf,
	font=small,
	textfont=it
}

% Diseño de figuras para ciencias sociales y humanidades
\captionsetup[figure]{
	name=Figura,
	singlelinecheck=off,
	labelsep=newline,
	font=small,
	labelfont=bf,
	textfont=it
}
\floatsetup[figure]{
    style=plaintop,
    heightadjust=caption,
    footposition=bottom,
    font=small
}

% Configuración del pie de las figuras y tablas 
\captionsetup*[floatfoot]{
    footfont={small, up}
}

% NOTAS A PIE DE PÁGINA
\usepackage{chngcntr} % para numeración continua de las notas al pie
\counterwithout{footnote}{chapter}

% LISTADOS DE CÓDIGO
% soporte y estilo para listados de código. Más información en https://es.wikibooks.org/wiki/Manual_de_LaTeX/Listados_de_código/Listados_con_listings
\usepackage{listings}

% definimos un estilo de listings
\lstdefinestyle{estilo}{ frame=Ltb,
	framerule=0pt,
	aboveskip=0.5cm,
	framextopmargin=3pt,
	framexbottommargin=3pt,
	framexleftmargin=0.4cm,
	framesep=0pt,
	rulesep=.4pt,
	backgroundcolor=\color{gray97},
	rulesepcolor=\color{black},
	%
	basicstyle=\ttfamily\footnotesize,
	keywordstyle=\bfseries,
	stringstyle=\ttfamily,
	showstringspaces = false,
	commentstyle=\color{gray45},     
	%
	numbers=left,
	numbersep=15pt,
	numberstyle=\tiny,
	numberfirstline = false,
	breaklines=true,
	xleftmargin=\parindent
}

\captionsetup*[lstlisting]{font=small, labelsep=period}
% fijamos el estilo a utilizar 
\lstset{style=estilo}
\renewcommand{\lstlistingname}{\uppercase{Código}}


%BIBLIOGRAFÍA 

% CONFIGURACIÓN PARA LA BIBLIOGRAFÍA APA
\usepackage[style=apa, backend=biber, natbib=true, hyperref=true, uniquelist=false, sortcites]{biblatex}
\DeclareLanguageMapping{spanish}{spanish-apa}

% Para sustituir % por y o e
\makeatletter
\DefineBibliographyExtras{spanish}{%
  \setcounter{smartand}{1}%
  \let\lbx@finalnamedelim=\lbx@es@smartand
  \let\lbx@finallistdelim=\lbx@es@smartand
}

% Si te sigue apareciendo "y" en vez de "e" cuando un apellido empieza por "I", puedes forzar que aparezca la "e" añadiendo esto antes del nombre del autor en el archivo referencias.bib: {\forceE{Inicial del nombre}}. Por ejemplo, Luis {\forceE{I}}añez debería aparecer "e Iañez, L." si es el último autor.

\renewbibmacro*{name:delim:apa:family-given}[1]{%
  \ifnumgreater{\value{listcount}}{\value{liststart}}
    {\ifboolexpr{
       test {\ifnumless{\value{listcount}}{\value{liststop}}}
       or
       test \ifmorenames
     }
       {\printdelim{multinamedelim}}
       {\lbx@finalnamedelim{#1}}}
    {}}
\makeatother


% Añadimos las siguientes indicaciones para mejorar la adaptación del estilos en español
\DefineBibliographyStrings{spanish}{%
	andothers = {et\addabbrvspace al\adddot}
}
\DefineBibliographyStrings{spanish}{
	url = {\adddot\space[En línea]\adddot\space Disponible en:}
}
\DefineBibliographyStrings{spanish}{
	urlseen = {Acceso:}
}
\DefineBibliographyStrings{spanish}{
	pages = {pp\adddot},
	page = {p.\adddot}
}

\addbibresource{referencias.bib} % llama al archivo referencias.bib en el que deberá estar la bibliografía utilizada


%-------------
%	DOCUMENTO
%-------------

\begin{document}
\pagenumbering{roman} % Se utilizan cifras romanas en la numeración de las páginas previas al cuerpo del trabajo
	
%----------
%	PORTADA
%----------	
\begin{titlepage}
	\begin{sffamily}
	\color{azulUC3M}
	\begin{center}
		\begin{figure}[H] %incluimos el logotipo de la Universidad
			\makebox[\textwidth][c]{\includegraphics[width=16cm]{logo_UC3M.png}}
		\end{figure}
		\vspace{2.5cm}
		\begin{Large}
			Máster Universitario...\\			
			 2020-2021\\ %Indica el curso académico
			\vspace{2cm}		
			\textsl{Trabajo Fin de Máster}
			\bigskip
			
		\end{Large}
		 	{\Huge ``Título del trabajo''}\\
		 	\vspace*{0.5cm}
	 		\rule{10.5cm}{0.1mm}\\
			\vspace*{0.9cm}
			{\LARGE Nombre Apellido1 Apellido2}\\ 
			\vspace*{1cm}
		\begin{Large}
			Tutor/es\\
			Nombre Apellido1 Apellido2\\
			Nombre Apellido1 Apellido2\\
			Lugar y fecha de presentación prevista\\
		\end{Large}
	\end{center}
	\vfill
	\color{black}
	\fbox{
		\begin{minipage}{\linewidth}
		\textbf{DETECCIÓN DEL PLAGIO}\\
		\footnotesize{La Universidad utiliza el programa \textbf{Turnitin Feedback Studio} para comparar la originalidad del trabajo entregado por cada estudiante con millones de recursos electrónicos y detecta aquellas partes del texto copiadas y pegadas. Copiar o plagiar en un TFM es considerado una \textbf{\underline{Falta Grave}}, y puede conllevar la expulsión definitiva de la Universidad.}\end{minipage}}
	
	% SI NUESTRO TRABAJO SE VA A PUBLICAR CON UNA LICENCIA CREATIVE COMMONS, INCLUIR ESTAS LÍNEAS. ES LA OPCIÓN RECOMENDADA.
	\noindent\includegraphics[width=4.2cm]{creativecommons.png}\\ %incluimos el logotipo de Creative Commons
	\footnotesize{Esta obra se encuentra sujeta a la licencia Creative Commons \textbf{Reconocimiento - No Comercial - Sin Obra Derivada}}
		
	\end{sffamily}
\end{titlepage}

\newpage %página en blanco o de cortesía
\thispagestyle{empty}
\mbox{}

%----------
%	RESUMEN Y PALABRAS CLAVE
%----------	
\renewcommand\abstractname{\large\bfseries\filcenter\uppercase{Resumen}}
\begin{abstract}
\thispagestyle{plain}
\setcounter{page}{3}
	
	% ESCRIBIR EL RESUMEN AQUÍ
	
	\textbf{Palabras clave:}
	% Escribir las palabras clave aquí
	
	\vfill
\end{abstract}
	\newpage % página en blanco o de cortesía
	\thispagestyle{empty}
	\mbox{}


%----------
%	DEDICATORIA
%----------	
\chapter*{Dedicatoria} % \chapter* evita que aparezca en el índice

\setcounter{page}{5}
	
	% ESCRIBIR LA DEDICATORIA AQUÍ	
		
	\vfill
	
	\newpage % página en blanco o de cortesía
	\thispagestyle{empty}
	\mbox{}
	

%----------
%	ÍNDICES
%----------	

%--
% Índice general
%-
\tableofcontents
\thispagestyle{fancy}

\newpage % página en blanco o de cortesía
\thispagestyle{empty}
\mbox{}

%--
% Índice de figuras. Si no se incluyen, comenta las líneas siguientes
%-
\listoffigures
\thispagestyle{fancy}

\newpage % página en blanco o de cortesía
\thispagestyle{empty}
\mbox{}

%--
% Índice de tablas. Si no se incluyen, comenta las líneas siguientes
%-
\listoftables
\thispagestyle{fancy}

\newpage % página en blanco o de cortesía
\thispagestyle{empty}
\mbox{}


%----------
%	MEMORIA
%----------	
\clearpage
\pagenumbering{arabic} % numeración con números arábigos para el resto de la memoria.	

\chapter{Introducción}

	% COMENZAR A ESCRIBIR la MEMORIA
	
	% IMPORTANTE: en LaTeX hay una serie de caracteres especiales, que son: # $ % & \ ^ _ { } ~. Si aparecen en el texto, tendrás un error al compilar. La mayoría se pueden escapar escribiendo \ delante. Para \ utiliza \textbackslash ; para ^ \textasciitilde y para ~ \textasciicircum.

    % COMO incluir una FIGURA siguiendo las recomendaciones de la Guía: Alineación del título: izquierda, en la parte superior de la tabla; fuente a 10pt (el resto de la memoria está a 12); Numeración de la figura en negrita, título de la figura en nueva línea en cursiva; Propiedad intelectual: Se debe indicar la fuente de origen de la información en la parte inferior de la figura, a continuación del título.
    
    % EJEMPLO DE INCLUSIÓN de una figura:
    % \begin{figure}[H]
    % 	\ffigbox[\FBwidth] {
    % 	\caption[Nombre que aparecerá en el índice]{Nombre que aparecerá sobre la figura}
    % 	\floatfoot{¿dónde hemos obtenido la imagen?}% P.e.: \floatfoot{Fuente: \textcite[p. 35]{id-referencia}}
    %     }
    % 	{\includegraphics[scale=0.6]{imagenes/creativecommons.png}}
    % \end{figure}
    
    % COMO incluir una TABLA siguiendo las recomendaciones de la Guía: Alineación del título: izquierda, en la parte superior de la tabla; fuente a 10pt (el resto de la memoria está a 12); Numeración de la figura en negrita, título de la figura en nueva línea en cursiva; Propiedad intelectual: Se debe indicar la fuente de origen de la información en la parte inferior de la tabla, a continuación del título.
    
    % EJEMPLO DE INCLUSIÓN de una tabla:
% \begin{table}[H]
% 	\ttabbox[\FBwidth]
% 	{\caption{Lorem ipsum}
% 	\floatfoot{Fuente: BOE}}
% 	{\begin{tabular}{|c|P{1.5cm}|c|P{1.5cm}|P{2cm}|c|P{1.5cm}|P{2cm}|}
% 		\hline
% 		\multicolumn{2}{|c|}{\textbf{I}} & \multicolumn{2}{c|}{\textbf{II}} & \multicolumn{3}{c|}{\textbf{III}} & \textbf{IV} \\
% 		\hline
% 		x & y & x & y & x & y & x & y \\
% 		\hline
% 		10.0 & 8.04 & 10.0 & 9.14 & 10.0 & 7.46 & 8.0 & 6.58 \\
% 		\hline
% 		8.0 & 6.95 & 8.0 & 8.14 & 8.0 & 6.77 & 8.0 & 5.76 \\
% 		\hline
% 		13.0 & 7.58 & 13.0 & 8.74 & 13.0 & 12.74 & 8.0 & 7.71 \\
% 		\hline
% 		9.0 & 8.81 & 9.0 & 8.77 & 9.0 & 7.11 & 8.0 & 8.84 \\
% 		\hline
% 		11.0 & 8.33 & 11.0 & 9.26 & 11.0 & 7.81 & 8.0 & 8.47 \\
% 		\hline
% 		14.0 & 9.96 & 14.0 & 8.10 & 14.0 & 8.84 & 8.0 & 7.04 \\
% 		\hline
% 		6.0 & 7.24 & 6.0 & 6.13 & 6.0 & 6.08 & 8.0 & 5.25 \\
% 		\hline
% 		4.0 & 4.26 & 4.0 & 3.10 & 4.0 & 5.39 & 19.0 & 12.50 \\
% 		\hline
% 		12.0 & 10.84 & 12.0 & 9.13 & 12.0 & 8.15 & 8.0 & 5.56 \\
% 		\hline
% 		7.0 & 4.82 & 7.0 & 7.26 & 7.0 & 6.42 & 8.0 & 7.91 \\
% 		\hline
% 		5.0 & 5.68 & 5.0 & 4.74 & 5.0 & 5.73 & 8.0 & 6.89 \\
% 		\hline
% 	\end{tabular}}
% \end{table}


%----------
%	BIBLIOGRAFÍA
%----------	

%\nocite{*} % Si quieres que aparezcan en la bibliografía todos los documentos que la componen (también los que no estén citados en el texto) descomenta está línea

\clearpage
\addcontentsline{toc}{chapter}{Bibliografía}
\setquotestyle[english]{british} % Cambiamos el tipo de cita porque en el estilo IEEE se usan las comillas inglesas.
\printbibliography



%----------
%	ANEXOS
%----------	

% Si tu trabajo incluye anexos, puedes descomentar las siguientes líneas
%\chapter* {Anexo x}
%\pagenumbering{gobble} % Las páginas de los anexos no se numeran



\end{document}